\section{Discussion and Conclusion}

A significant side benefit of the more sensitive sensors (LMS-200, Hokuyo UTM-30LX-EW) is that they are capable of recording the temporal evolution of a snowstorm, at a fine-grained level. This is something that is not possible with traditional snow measuring equipment, which can only report accumulation over long period of times. Indeed, for b) we see a pretty steady snowfall from $t=1$ to $t=4$ hours, with a peak around $t=1.25$ hours. This could be helpful in developing temporal models of snowstorms, to be used in a vehicle simulator for example or simply in meteorological studies.

Open question such as how the maximum range is affected by the snow.

Say we will do more analysis.
We have more dataset, including sunny days (to see the impact of the snow reflection ambient light) and one rainy episode. Also look at how noise was affected.

How the snow seems only to affect at close range. due to the physics of the problem : reduced beam intensity, reduced echo.
