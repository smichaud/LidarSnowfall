%!TEX root = ../root.tex
\begin{abstract}
Autonomous driving vehicles must be able to handle difficult weather conditions in order to gain acceptance. For example, challenging situations such as falling snow could significantly affect the performance of vision or LiDAR-based perception systems. In this paper, we are interested in characterizing the behavior of LiDARs in snowy conditions, as there seems to be little information publicly available. In particular, we present a characterization of the behavior of 4 commonly-used LiDARs (Velodyne HDL-32E, SICK LMS151, SICK LMS200 and Hokuyo UTM-30LX-EW) during the falling snow condition. Data was collected from the 4 sensors simultaneously during 10 snowfalls. Statistical analysis of these data sets indicates that these sensors can be modeled in a probabilistic manner, allowing the use of a Bayesian framework to improve robustness. Using data provided by the multi-echo LiDAR UTM-30LX-EW, we analyze the temporal evolution of snowstorms, in order to replicate their general behavior in simulation. 
\end{abstract}


