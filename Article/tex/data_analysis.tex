%!TEX root = ../root.tex
\section{Data analysis}
\label{sec:data-analysis}

\todo{I suggest a reorganization of the sections. 1) put ``pre-selection of laser data'' at the end of the previous section. ``temporal analysis'' and ``range analysis'' become their own sections.}

\todo{If not, then we need an intro laying down the plan: first discuss dataset selection, then temporal analysis, followed by range analysis to help the reader figure out what's going on. }

\subsection{Pre-selection of laser data}
For each sensor, we selected a combination of angles and laser rings (for the Velodyne) or beams (for the others) that had a clear view of the snow-covered ground surface. The actual details for each sensor are given in tab.~\ref{tab:selectionScans}. The range of the ground in our scans was between $x = \SI{15}{\meter}$ to $x=\SI{22}{\meter}$, depending on the beam. To simplify the analysis, we considered as a snowflake echo any measurement which had a range reading of $x<\SI{14.5}{\meter}$. As will be shown later in sec.~\ref{subsub:Histo}, this is a valid approximation as the vast majority of those events happened for $x<\SI{10}{\meter}$. Thus, any snowflakes echoes between $x>\SI{14.5}{\meter}$ and the snow-covered ground surface were negligible, for all four sensors.

\begin{table}[htbp]
    \centering
    % \def\tabularxcolumn#1{m{#1}}
    \begin{tabular}{|c|c|c|c|c|}
        \hline
        \textbf{Sensor}            & \textbf{Acquisition}  & \textbf{Selected}  & \textbf{Selected}  & \textbf{Window} \\
        \textbf{ }                     & \textbf{frequency}  & \textbf{beams/angles}  & \textbf{rings}  & \textbf{size} \\\hline
       LMS-200               & \SI{9.375}{\Hz}                      & 55--115                                    & N/A                         & ~\SI{106}{\second}       \\\hline
        LMS-151               & \SI{25}{\Hz}                           & 310--220                                  & N/A                         & ~\SI{40}{\second}        \\\hline
        Hokuyo               & \SI{20}{\Hz}                          & 440--590                                  & N/A                         & ~\SI{100}{\second}     \\\hline
        Velodyne             & \SI{10}{\Hz}                          & -0.05--0.25 rad                     & 17--31                   & ~\SI{40}{\second}      \\\hline
    \end{tabular}
    \caption{Details for data used in analysis. The window size is the temporal window used to calculate statistics during the temporal evolution of a storm.}
    \label{tab:selectionScans}
\end{table}

% ========================= Timing  ===================

\subsection{Temporal analysis}
Snowstorms are highly dynamic processes, with large variation in snowfall rates over their durations. Moreover, the snow physical characteristics (size, shape or reflectance) might vary significantly during a storm, affected by ambient conditions such as humidity level and temperature. Also, wind gusts might pull snow back up in the air or drive it sideways, affecting its effective fall rate. Consequently, one expects during a snowstorm to see significant short, medium and long term variations in the fraction of LiDAR echoes corresponding to the falling snow. 

Computing and reporting the temporal statistics for every scan would put too much emphasis on the very short-term statistics. Indeed, the inter-scan variation in the fraction of snowflake echoes can be significant. To better illustrate this point, we have overlaid four consecutive scans in the same plot for the LMS-200 and for the first echo returned by the multi-echo Hokuyo sensor in fig.~\ref{fig:LMS200_4Scans_Feb19}, for an intense snowing episode from the 02-19 data set (see tab.~\ref{tab:overview-dataset}). In these figures, we can see both the spatial and temporal variations of the echoes in snowflakes, which we suspect can be best described by a random process. For the discussion at hand, one can see a fluctuation in the fraction of echoes in snowflakes between scans, as reported in the brackets of the legend. 

To smooth out these very short-term fluctuations, statistics are extracted from a number of consecutive scans contained in a time window of around 1~minute (detailed values in tab.~\ref{tab:selectionScans}~\todo{why these values of 106, 40, and 100 s?}). Fig.~\ref{fig:TimingSnow}, shows this smoothed fraction of snowflakes echos compared to all returned laser measurements as a function of time, for the six snowiest days of our dataset. To allow for better visualization, only the LMS-200 and the Hokuyo's first echo are plotted at their actual scale (1x): Others have been scaled up (from 30x to 200x), with their corresponding scaling factors reported in the legend. 

 \begin{figure}[th]
    \centering
    \includegraphics[width=0.95\linewidth]{./img/LMS200_4Scans_Feb19.png}
    \includegraphics[width=0.95\linewidth]{./img/Hokuyo_4Scans_Feb19.png}
    \caption{Four overlaid consecutive scans for the LMS200 sensor (top), and the first echo scans for the Hokuyo sensor (bottom), taken from the 02-19 dataset. Each symbol correspond to a particular scan. The curved line at the top corresponds to the snow surface on the ground. One can see the rapid variation of the snowflake echoes between scans, and how they are mostly limited to a range $x<\SI{5}{\meter}$. The percentage (in brackets) are the proportion of those echoes in the snowflakes. \todo{use PDF instead}}
    \label{fig:LMS200_4Scans_Feb19}
\end{figure}

\subsubsection{SICK Sensors LMS-200 and LMS-151}
Our first conclusion based on fig.~\ref{fig:TimingSnow} is that some sensors are more sensitive than others. The most sensitive device was the older LMS200, first introduced in the mid-2000s. For the most intense snowstorms (fig.~\ref{fig:TimingSnow}. b) 02-19, d) 03-17, e) 03-21 and f) 03-30), it peaked at around 15\%, for averaging windows of \SI{106}{\second}. As an older-generation device, it probably used unsophisticated algorithms and sensing. Indeed, its technical description~\cite{LMS200Manual} indicates that ``Raindrops and snow-flakes are cut out using pixel-oriented evaluation'', but this seems only applicable to  obstacle detection (field computation), not the actual measurements. However, no further details are given. On the other hand, the much more recent SICK LMS-151 exhibits much less sensitivity to snowflakes: the reduction factor for the fraction of snowflakes echoes is in the order of 200-300, granting this device a much higher immunity against snowflakes. Indeed, the highest peak was around 0.1 \% during the 02-19 dataset. Advances in optics and algorithms are probably responsible for this significant improvement~\todo{sounds too generic and hypothetical. Remove?}. Moreover, this device can be programmed to return either the first or second echo\todo{Make sure this information is ok.}. In our tests, we selected the latter: it would have been however interesting to perform tests to get information about the first echo, but our current system was not capable of gathering this information. Finally, this sensor can also do optics cover contaminant measurements, albeit this feature was not relevant in our case since the sensor's cover was not exposed to the elements~\todo{also remove? not super relevant}.

\begin{figure*}[th]
    \centering
    \includegraphics[width=0.98\linewidth]{./img/TimingSnow.png}
    \caption{Temporal evolution of the percentage of echoes coming from the falling snow (range $x<$\SI{5}{\meter}) during the 6 most intense episodes, for all 4 sensors. The data is smoothed by taking statistics for small time windows. Except for the LMS-200 and Hokuyo first echo, all other sensors statistics have been scaled up (factor in bracket of legend b) for ease of visual comparison. Time is in hour, starting from the beginning of the data capture sequence. \todo{Replace with PDF version?}}
    \label{fig:TimingSnow}
\end{figure*}


\subsubsection{Hokuyo UTM-30LX-EW}
For the Hokuyo sensor, we resorted to a slightly different approach for comparison, as the device has been designed to return multiple echoes. We thus extracted statistics for two cases: for first and last echo. Statistics for the first echo tells us how sensitive the device is, if one wishes to detect the presence or absence of falling snow. This information could be used, in turn, to adapt the driving or inform vision algorithms of the presence of particles in the air. Using the last echo ensures that we are able to detect large obstacles such as other vehicle or the snow-covered ground, and used in localization and navigation purposes. In case of the first echo, the device should behave like the LMS-200, and in the second case like the LMS-151 (recall that it was programmed to return the second echo)~\todo{why?? statement not grounded. consider removing?}. When looking at fig.~\ref{fig:TimingSnow}, this is more-or-less the behavior that we see~\todo{vague---needs rephrasing}. The Hokuyo first echo (blue line) closely tracks the LMS-200 curves (red dashed line) almost everywhere, with a few exceptions for which we do not have an explanation~\todo{vague!}. The last echo of the Hokuyo tends to reject the falling snow, but not as well as the LMS-151, as it peaked at around 0.5 \% in some episodes. Tab.~\ref{tab:avgRates} shows a similar correlation, but for averages taken over the complete 02-19 dataset. Nevertheless, this difference might not be sufficient to impact algorithms relying on laser data.

\subsubsection{Velodyne HDL-32E}
For all purposes, the behavior of the Velodyne was similar to the last echo of the Hokuyo sensor. This is seen both in the temporal behavior in fig.~\ref{fig:TimingSnow} than in the average value displayed in tab.~\ref{tab:avgRates}.

% ========================= Histograms ===================
\subsection{Distribution of echoes in snowflakes, as a function of range $x$}
\label{subsub:Histo}

\subsubsection{Modeling}
When modeling a range sensor, one has to have an idea of the probability distribution of certain events (e.g. snowflakes) as a function of the distance to the sensor. Over the years, many researchers have proposed probabilistic models for sensors, notably in Thrun et al.~\cite{Thrun:2005:PR:1121596}. In the previous section, we have in some sense estimated the probability for a given sensor S that a snowflake would generate an echo $E_\text{snowflake}$ given the weather condition $W$, or $P_S(E_\text{snowflake}|W)$. In this section, we take a closer look at which range $x$ such events would be generated, $P_S(E_\text{snowflake}|x,W)$. Having such a formulation would allow for a more statistically-sound treatment of the information, such as within the Bayesian probabilistic framework. To this effect, we use histograms as approximations to the previous distribution. In fig.~\ref{fig:Histograms}, we have plotted these histograms for each of the four sensors. For ease of comparison, these histograms have all been normalized by their total area in the interval $0 < x < \SI{14}{\meter}$, as the total count varies widely between the sensors. The numbers in brackets in the legend indicate the fraction of echoes in the snowflakes compared to the total number of data points, for a particular dataset.

%, such as
%\begin{equation}
%p(e|x,w)=p(e|w)p(e|x)
%\end{equation}
%where p(e|x) would be the log-normal distribution and p(e|w) the probability of having snow affecting

The general shape of these histograms is close to a log-normal distribution, with the exception of the LMS-200 for a number of dates (02-12 through 03-17), which seems to follow a sum of two log-normal distributions. We attribute this shape to two different phenomenon, illustrated in a cartoon-type model in fig.~\ref{fig:CartoonModel}. At short ranges $x<\SI{3}{\meter}$, the building acts as a shield and decreases the probability of having a snowflake in the path of the laser. This phenomenon would be absent, to some extent, on an autonomous vehicle. For this reason, we prefer not to discuss it too much~\todo{vague. remove?}. The other phenomenon, illustrated as the red dashed line in fig.~\ref{fig:CartoonModel}, is the probability of optical detection of a snowflake by the sensor as a function of the range $x$. We argue that this shape is due to the rapidly decreasing light intensity of the echoes as a function of the distance $x$. Combining these two phenomenon yields a log-normal shaped curve (black line in fig.~\ref{fig:CartoonModel}). Overall, this seems to indicate that a simple probabilistic model $P_S(E_\text{snowflake}|x,W)$ can be derived for these sensors. 

% of the echoes would exhibit a curve similar to t, coupled with the decreased light flux/intensity if one assume constant beam size on the way out. These two factors are are  Again, because this is not something we expect to see on a real vehicle, we will not spend too much time on this model. (1/$x^2$) 

\subsubsection{Sensor results}
As can be seen from fig.~\ref{fig:Histograms}, most sensors exhibit the log-normal or sum-of-log-normal distributions discussed above. We note also that for certain days, the distributions are shifted to the right (higher ranges $x$). In particular, for the 03-21 and the 03-30 distributions, this shift is substantial (in the order of \SI{1}{\meter}). We suspect that for these particular days, the snowflakes were significantly larger, thus allowing for a stronger optical echo and extended range of detection. 

For all sensors, we can also conclude that beyond the range $x>\SI{10}{\meter}$, snowflakes are no longer detected. A small notable exception would be for the Velodyne, for which snowflakes were detected all the way to $x=\SI{14}{\meter}$, albeit at a significantly reduced rate. Again, we do not think that this would significantly impair their use in conditions similar to our test setup. 

%Note that for the LMS-151, the amount of data was not sufficient to draw much conclusion on this distribution except for the fact that there was virtually no events passed 4.

\begin{figure}[th]
    \centering
    \includegraphics[width=0.80\linewidth]{./img/Histograms.png}
    \caption{Histograms of echoes in falling snow during important snowfall days, as a function of distance $x$ reported by the sensor. Each histogram has been normalized by its area, for ease of comparison. The numbers in brackets are the fraction of data points in the complete data set that correspond to snowflake echoes. Note that for the 03-21 dataset, the LMS151 was not working properly: thus no data is included for that day. \todo{PDF}}
    \label{fig:Histograms}
\end{figure}

\begin{figure}[th]
    \centering
    \includegraphics[width=0.97\linewidth]{./img/ShieldingModel.png}
    \caption{Cartoon representation of the interaction between the probability of detecting a snowflake (in red) and the diminution of snowflakes due to the shielding effect of the building (in blue). The black line is the product of the two, and bear a close resemblance to the actual histograms extracted from our data sets. \todo{PDF}}
    \label{fig:CartoonModel}
\end{figure}


\begin{table}[htbp]
    \centering
    \todo{Might want to give info about the why values are different from fig 4.}
    \begin{tabular}{|c|c|c|c|c|}
        \hline
        \textbf{LMS-200}       & \textbf{Hokuyo}             & \textbf{Hokuyo}    & \textbf{LMS-151}  & \textbf{Velodyne}  \\
                                        & \textbf{first echo}   & \textbf{last echo}  &                            & \textbf{HDL-32E}  \\\hline
                 2.67\%            &           3.55\%                &       0.0113\%      &       0.00178\%     &  0.0100\%  \\\hline
    \end{tabular}
    \caption{Average snowflakes echos for the 02-19 data set, per sensor.}
    \label{tab:avgRates}
\end{table}

% ======================= Sunlight =====================
%\subsection{Impact of sunlight on Hokuyo}
%During our testing, we noticed that the Hokuyo sensor seems sentitive to sunlight. We collected a data set during a sunny day (02-16), to better illustrate this condition. 

% Note that however there was instance (daytime with sun) where the sensor didn't return echoes at all. We suspect that this is sue to the fact that the power level used. More on this in a following subsection.

% ========================================
%\section{Discussion}
%It seems that the majority of the impact of falling snow is for the first 5-6 meters. For near distance, this would impact the collision/obstacle avoidance. However, the problem at long range does not seem to happen, at least based on extrapolating from our data set.



